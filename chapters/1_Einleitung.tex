\chapter{Einleitung}
Diese Arbeit dreht sich um die Analyse von Java-Systemen, sowie der Entwicklung eines Algorithmus zur Generierung einer verbesserten Softwarearchitektur. Zum Auslesen der Metadaten des Java-Systems wird das  Analysewerkzeug jQAssistant verwendet. Die Analyse der Metadaten und der Algorithmus basieren auf den Standards von Domian-Driven Design.

In diesem Kapitel wird die Motivation hinter dieser Arbeit betrachtet und es wird auf die daraus folgenden Problemstellungen eingegangen. Außerdem wird das Ziel dieser Arbeit näher betrachtet und anschließend der Aufbau der Arbeit erläutert.

\section{Motivation}
Heutzutage wird häufig, beim Entwickeln von neuer Software, auf die Standards von Domain-Driven Design zurückgegriffen. Besonders häufig findet es seinen Einsatz in der Entwicklung von Microservices \cite{DDD:Rademacher:2018a, DDD:Rademacher:2018b}. Aber bereits existierenden Systeme erfüllen nicht immer alle Standards vom Domain-Driven Design. Dadurch leidet meistens die Codequalität. Domain-Driven Design kann zum Beispiel dabei helfen modularen Quellcode zu entwickeln, wodurch das System leichter verändert werden kann. Außerdem sorgen die Standards dafür, dass der Quellcode lesbarer wird und weniger Duplikate enthält. 

Aufgrund dieser Vorteile kann es nützlich sein die Architektur eines vorhandenen Systems auf Basis von Domain-Driven Design abzuändern, beziehungsweise die Einhaltung der Standards von Domain-Driven Design zu erhöhen. Das Refactoring kann allerdings sehr viel Zeit und Aufwand in Anspruch nehmen, weshalb eine Automatisierung dieses Vorgangs sehr hilfreich sein kann.

\section{Problemstellung}
Bei dem Vorhaben ein vorhandenes System, auf Basis von Domain-Driven Design, abzuändern, müssen folgende Problemstellungen beachtet werden. 

Zuerst stellt sich die Frage, wie man die Architektur eines Java-Systems analysieren kann. Hierfür bietet sich das Analysewerkzeug jQAssistant an, da es dem Nutzer ermöglicht eine Strukturanalyse von Java-Artefakten durchzuführen. Allerdings eignet sich jQAssistant nicht für alle Anwendungsfälle. Deshalb müssen die Grenzen von jQAssistant bei der Analyse von Java-Artefakten herausgearbeitet werden.

Abgesehen von der Auswahl eines Analysewerkzeugs muss geklärt werden anhand von welchen Merkmalen Domain-Driven Design erkannt werden kann und zudem müssen die Grenzen des erzeugten Verbesserungsvorschlags definiert werden.

\section{Ziel der Arbeit}

\begin{figure}
    \centering
    \includegraphics[width=1.0\textwidth]{gfx/illumi-code-ddd-workflow-German.pdf}
    \caption{Ablaufplan dieser Arbeit}
    \label{fig:Workflow}
\end{figure}

Mit der Arbeit soll ein Konzept entwickelt werden, mithilfe dessen überprüft werden kann, ob ein bestehendes Java-System die Prinzipien von Domain-Driven Design einhält. 

Wie man in der Abbildung \ref{fig:Workflow} erkennen kann, soll zunächst ein bestehendes Java-System mit jQAssistant analysiert werden und anschließend anhand einer Kennzahl, zum Beispiel die prozentuale Einhaltung der Prinzipien, bewertet werden. Außerdem soll ein Algorithmus entwickelt werden, welcher die nicht eingehaltenen Prinzipien, nach Möglichkeit, verbessert. Dabei liegt der Schwerpunkt auf der Verbesserung der Architektur. Anschließend soll der Verbesserungsvorschlags erneut bewertet werden. Der Algorithmus gilt als erfolgreich, wenn die Bewertung des verbesserten Systems besser geworden ist oder sich nicht verschlechtert hat.

\section{Gliederung}
Die Arbeit ist wie folgt aufgebaut. Nach dieser Einleitung wird ausführlich auf die Grundlagen dieser Arbeit eingegangen. Dieser Teil unterteilt sich in eine Einführung in Domain-Driven Design und in das Analysewerkzeug jQAssistant. Anschließend werden verwandte Arbeiten und der aktuelle Stand der Technik betrachtet. Darauf folgt der Hauptteil dieser Arbeit. Der Hauptteil gliedert sich in zwei Kapitel. Im ersten Kapitel wird die Analyse des Java-Systems betrachtet. Dabei wird besonders ein Schwerpunkt auf die Erkennung der Merkmale von Domain-Driven Design gelegt. Das zweite Kapitel des Hauptteils beschäftigt sich mit der Umsetzung des Umwandlungsalgorithmus. Die Arbeit endet mit einer Diskussion der Ergebnisse sowie einer Zusammenfassung und einem kurzen Ausblick auf zukünftige Forschung in diesem Bereich.