\chapter{Grundlagen}


\section{Einführung in Domain-Driven Design}
\cite{DDD:Evans:2003}
\cite{DDD:Evans:2014, DDD:Plod:2019}
\cite{DDD:Vernon:2013}
\cite{DDD:Vernon:2016}
\cite{DDD:Vernon:2017}
\cite{DDD:Avram:2007}
\cite{DDD:Rademacher:2018b}


\section{Einführung in jQAssistent}
Zur Analyse des bestehenden Java-Systems wird das Analyse-Werkzeug jQAssistant verwendet. Das Ziel von jQAssistant ist \glqq die Zerlegung eines komplexen Ganzen in eine Menge beherrschbarer Einheiten und die Definition der Beziehungen zwischen diesen.\grqq{}\cite{jQA:Mahler:2015} Dafür bietet jQAssistant die Möglichkeit Java-Systeme auf struktureller Ebene, anhand von projektspezifische Regeln, zu analysieren \cite{jQA:Buschmais:2019}.

jQAssistant akzeptiert die Java-Projekte in den drei üblichen Formaten Java Archive (JAR), Enterprise Archive (EAR) und Web Application Archive (WAR). Beim Analysieren werden verschiedene jQAssistant Plugins ausgeführt. Jedes Plugin besteht aus zwei Teilen, dem Scanner und Regeln. Der Scanner extrahiert die Daten aus dem Java-Artefakt und speicher diese in der Neo4j Graphdatenbank. Anschließende werden auf diese Daten verschiedenen Regeln zur Datenanreicherung ausgeführt.\cite{jQA:Muller:2018}

Die Daten in der Neo4j-Datenbank sind als Knoten modelliert. Jeder Knoten verfügt über Attribute, welche die Eigenschaften des Knotens beschreiben, zudem sind die einzelnen Konten mit Kanten verbunden welche die Beziehungen modellieren \cite{jQA:Muller:2018, jQA:Mahler:2015}.

\begin{lstlisting}[caption={Cypher-Abfrage zum filtern aller Klassen in einem bestimmten Package},label=lst:cypher]
    MATCH (package:Package)-[:CONTAINS]->(file:File) 
    WHERE package.fqn = "org.junit" AND file:Class
    RETURN file as Klassen
\end{lstlisting}

Auf diese Analysedaten kann mithilfe der Abfragesprache Cypher zugegriffen werden \cite{jQA:Muller:2018}. In \ref{lst:cypher} ist eine solche Abfrage zu sehen. Wie man an dieser Anfrage sieht, besteht eine Cypher-Abfrage auf drei Bausteinen. In der ersten Zeile wird definiert welche Knoten, mit welcher Beziehungen, berücksichtigt werden sollen. In diesem Fall sind alle Packages intressant welche einen oder mehrere Datenen beinhalten. In der zweiten Zeile wird näher spezifiziert welche Packages und welche Dateien relevant sind. In diesem Beispiel soll nur das Package mit dem Fully Qualified Name (FQN) \glqq org.junit\grqq{} in Verbindung mit Java-Klassen berücksichtigt werden. Die letzte Zeile der Abfrage gibt anschließend an, was zurückgegeben werden soll, in diesem Fall alle Klassen die im Package \glqq org.junit\grqq{} enthalten sind.

