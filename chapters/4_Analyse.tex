\chapter{Analyse der Architektur eines Java Systems}
Diese Kapitel behandelt die Umsetzung der Architekturanlyse in Verbindung mit der Entwicklung einer Metrik zur Architekturbewertung. Im ersten Schritt wird darauf eingegangen anhand welcher Merkmale Domain-Driven Design Bausteine erkannt wurden. Danach wird der Schwerpunkt auf die Bewertung der eingelesenen Bausteine sowie der zugehörigen Architektur gelegt. Als letztes werden die einzelnen Bewertungen zu einer Metrik zusammengefasst, um den Zustand des eingelesenen Systems zu bewerten.

\section{Erkennung der Merkmale von Domain-Driven Design}
\cite{DDD:Vernon:2013}
\cite{DDD:Vernon:2017}

\cite{DDD:Holmstrom:2019}
\cite{DDD:Le:2016}
\cite{DDD:Rademacher:2018a}
\cite{DDD:Hippchen:2017}
\cite{DDD:Nair:2019}


\section{Bewertung der Domain-Driven Design Artefakte}
In diesem Abschnitt wird darauf eingegangen anhand von welchen Kriterien die Artefakte bewertet werden. Ein Kriterium besteht aus zwei Bestandteilen. Zum einem aus einer Beschreibung was erfüllt sein muss, um das Kriterium zu erfüllen. Der zweite Bestandteil ist die Relevanz des Kriteriums. Die Relevanz hat einen direkten Einfluss auf die Gewichtung des Kriteriums.

\begin{table}[ht]
    \centering
    \begin{tabular}{|l||c|c|c|c|c|}   \hline
        \textbf{Einstufungen}   & INFO  & MINOR & MAJOR & CRITICAL  & BLOCKER   \\ \hline
        \textbf{Gewichtung}     & 0     & 1     & 3     & 5         & 10        \\ \hline
    \end{tabular}
    \caption{Einstufungen der Relevanz eines Kriteriums}
    \label{tab:criteria}
\end{table}

In der Tabelle \ref{tab:criteria} sind die fünf Einstufungen der Relevanz aufgeführt. Jeder Einstufung ist eine Gewichtung zugeordnet. Zum Beispiel sind Kriterien die als INFO gekennzeichnet sind nicht relevant für die Bewertung. Diese Kriterien beschreiben häufig die Benennung von Komponenten. 

Als MINOR oder MAJOR eingestufte Kriterien sollten eingehalten werden, aber von ihrer Einhaltung hängt nicht die komplette Architektur ab. Dies betrifft zum Beispiel das implementieren von vorgegebenen Methoden.

Kriterien welche hingegen als CRITICAL oder BLOCKER gekennzeichnet wurden, sollten zwingend erfüllt sein, da sonst Grundprinzipien der Architektur verletzt werden. Deshalb werden diese Kriterien besonders stark gewichtet.

\subsection{Bewertung von Modulen}
Bei der Bewertung von Modulen muss man zwischen drei Kategorien von Modulen unterscheiden. Das wichtigste Modul im Domain-Driven Design ist das Domain-Modul. Denn im Domain-Modul sind alle Domain-Artefakte enthalten. Das zweit wichtigste Modul ist das Application-Modul. Das Application-Modul beinhaltet nahezu die gesamte Geschäftslogik des Systems. Das dritte Modul ist das Infrastruktur-Modul. In diesem Modul sind Schnittstellen implementiert, zum Beispiel für Datenbanken oder Benutzeroberflächen.

\begin{table}[ht]
    \centering
    \begin{tabular}{|m{0.8\textwidth}|c|}   \hline
        \textbf{Kriterien}                                          & \textbf{Relevanz} \\ \hline\hline
        Submodul von \glqq domain\grqq{}                            & MINOR             \\ \hline
        Beinhaltet nur Domain-Artifakte                             & MAJOR             \\ \hline
        Beinhaltet Aggregate Root                                   & CRITICAL          \\ \hline
    \end{tabular}
    \caption{Kriterien zur Bewertung von Domain-Modulen}
    \label{tab:criteria:Module:Domain}
\end{table}

In der Tabelle \ref{tab:criteria:Module:Domain} sind die Kriterien von Domain-Modulen aufgelistet. Das erste Kriterium beschreibt, dass jedes Domain-Modul dem Modul \glqq domain\grqq{} untergeordnet sein sollte. Dieses Kriterium wurde als MINOR eingestuft, da es als Best Practice gilt aber nicht zwingend notwendig ist. 

Das zweite Kriterium betrachtet den Inhalt des Moduls, denn in einem Domain-Modul sollten nur Domain-Artefakte enthalten sein und keine Geschäfts- oder Infrastrukturlogik. Da dieses Kriterium wichtiger ist als das erste wurde es als MAJOR gekennzeichnet.

Das letzte Kriterium des Domain-Moduls, aus der Tabelle \ref{tab:criteria:Module:Domain}, ist gleichzeitig das wichtigste. Denn jedes Domain-Modul symbolisiert einen Domain und muss daher genau eine Aggregate Root beinhalten. Da diese Kriterium für die korrekte Umsetzung von Domain-Driven Design notwendig ist wurde es als CRITICAL eingestuft.

\begin{table}[ht]
    \centering
    \begin{tabular}{|m{0.8\textwidth}|c|}   \hline
        \textbf{Kriterien}                                          & \textbf{Relevanz} \\ \hline\hline
        Submodul von \glqq application\grqq{}                       & MINOR             \\ \hline
        Beinhaltet nur Application-Artifakte                        & MAJOR             \\ \hline
    \end{tabular}
    \caption{Kriterien zur Bewertung von Application-Modulen}
    \label{tab:criteria:Module:Application}
\end{table}

\begin{table}[ht]
    \centering
    \begin{tabular}{|m{0.8\textwidth}|c|}   \hline
        \textbf{Kriterien}                                          & \textbf{Relevanz} \\ \hline\hline
        Submodul von \glqq infrastructure\grqq{}                    & MINOR             \\ \hline
        Beinhaltet nur Infrastructure-Artifakte                     & MAJOR             \\ \hline
    \end{tabular}
    \caption{Kriterien zur Bewertung von Infrastruktur-Modulen}
    \label{tab:criteria:Module:Infrastructure}
\end{table}

Die Kriterien für Application- und Infrastruktur-Module, aus den Tabellen \ref{tab:criteria:Module:Application} und \ref{tab:criteria:Module:Infrastructure}, sind den ersten zwei Kriterien des Domain-Moduls sehr ähnlich.

\subsection{Bewertung von Domain Artefakten}
\begin{table}[ht]
    \centering
    \begin{tabular}{|m{0.8\textwidth}|c|}   \hline
        \textbf{Kriterien}                                          & \textbf{Relevanz} \\ \hline\hline
        Modul: \glqq domain.<domain>.model\grqq{}                   & MINOR             \\ \hline
        Attribut-Typen: Entities oder Value Objects                 & MAJOR             \\ \hline
        Ein Attribut für die eigene Identität                       & CRITICAL          \\ \hline
        Methoden equals und hashCode                                & MINOR             \\ \hline
        Getter für jedes Attribut                                   & MINOR             \\ \hline
        Setter für jedes Attribut                                   & MINOR             \\ \hline
    \end{tabular}
    \caption{Kriterien zur Bewertung von Entities}
    \label{tab:criteria:Entity}
\end{table}

 
\begin{table}[ht]
    \centering
    \begin{tabular}{|m{0.8\textwidth}|c|}   \hline
        \textbf{Kriterien}                                          & \textbf{Relevanz} \\ \hline\hline
        Modul: \glqq domain.<domain>.model\grqq{}                   & MINOR             \\ \hline
        Attribut-Typen: Value Objects oder Standard Datentypen      & MAJOR             \\ \hline
        Kein Attribut für eigenen Identität                         & BLOCKER           \\ \hline
        Methoden equals und hashCode                                & MINOR             \\ \hline
        Getter für jedes Attribut                                   & MINOR             \\ \hline
        Getter heißen so wie das zugehörige Attribut                & INFO              \\ \hline
        Setter für jedes Attribut                                   & MINOR             \\ \hline
        Setter ist frei von Seiteneffekten.                         & CRITICAL          \\ \hline
    \end{tabular}
    \caption{Kriterien zur Bewertung von Value Objects}
    \label{tab:criteria:ValueObject}
\end{table}
 
\begin{table}[ht]
    \centering
    \begin{tabular}{|m{0.8\textwidth}|c|}   \hline
        \textbf{Kriterien}                                          & \textbf{Relevanz} \\ \hline\hline
        Modul: \glqq domain.<domain>.model\grqq{}                   & MINOR             \\ \hline
        Attribut-Typen: andere Entities oder Value Objects          & MAJOR             \\ \hline
        Ein Attribut für die eigene Identität                       & CRITICAL          \\ \hline
        Methoden equals und hashCode                                & MINOR             \\ \hline
        Getter für jedes Attribut                                   & MINOR             \\ \hline
        Setter für jedes Attribut                                   & MINOR             \\ \hline
        Repository für die Aggregate Root existiert                 & MAJOR             \\ \hline
        Factory für die Aggregate Root existiert                    & MAJOR             \\ \hline
        Service für die Aggregate Root existiert                    & MAJOR             \\ \hline
    \end{tabular}
    \caption{Kriterien zur Bewertung von Aggregate Roots}
    \label{tab:criteria:AggregateRoot}
\end{table}
 
\begin{table}[ht]
    \centering
    \begin{tabular}{|m{0.8\textwidth}|c|}   \hline
        \textbf{Kriterien}                                          & \textbf{Relevanz} \\ \hline\hline
        Modul: \glqq domain.<domain>.model\grqq{}                   & MINOR             \\ \hline
        Attribut für Zeitstempel                                    & MAJOR             \\ \hline
        Attribut für die ID einer Entity                            & MAJOR             \\ \hline
    \end{tabular}
    \caption{Kriterien zur Bewertung von Domain Events}
    \label{tab:criteria:DomainEvent}
\end{table}

\begin{table}[ht]
    \centering
    \begin{tabular}{|m{0.8\textwidth}|c|}   \hline
        \textbf{Kriterien}                                          & \textbf{Relevanz} \\ \hline\hline
        Interface-Namenssuffix: \glqq Repository\grqq{}             & INFO              \\ \hline 
        Klassen-Namenssuffix: \glqq RepositoryImpl\grqq{}           & INFO              \\ \hline
        Interface-Modul: \glqq domain.<domain>.model\grqq{}         & MINOR             \\ \hline
        Klassen-Modul: \glqq domain.<domain>.model.impl\grqq{}      & MINOR             \\ \hline
        Repository-Klasse implementiert Repository-Interface        & MINOR             \\ \hline
        Methode zum Erzeugen einer neuen Identität (nextIdentity)   & MAJOR             \\ \hline
        Methode zum Finden (findBy/get)                             & MAJOR             \\ \hline
        Methode zum Speichern (save/add/insert/put)                 & MAJOR             \\ \hline
        Methode zum Löschen (delete/remove)                         & MAJOR             \\ \hline
        Methode zum Überprüfen der Existenz (contains/exists)       & MINOR             \\ \hline
        Methode zum Aktualisieren (update)                          & MINOR             \\ \hline
    \end{tabular}
    \caption{Kriterien zur Bewertung von Repositories}
    \label{tab:criteria:Repository}
\end{table}

\begin{table}[ht]
    \centering
    \begin{tabular}{|m{0.8\textwidth}|c|}   \hline
        \textbf{Kriterien}                                          & \textbf{Relevanz} \\ \hline\hline
        Interface-Namenssuffix: \glqq Factory\grqq{}                & INFO              \\ \hline 
        Klassen-Namenssuffix: \glqq FactoryImpl\grqq{}              & INFO              \\ \hline
        Interface-Modul: \glqq domain.<domain>.model\grqq{}         & MINOR             \\ \hline
        Klassen-Modul: \glqq domain.<domain>.model.impl\grqq{}      & MINOR             \\ \hline
        Factory-Klasse implementiert Factory-Interface              & MINOR             \\ \hline
        Methode zum Erzeugen (create)                               & MAJOR             \\ \hline
    \end{tabular}
    \caption{Kriterien zur Bewertung von Factories}
    \label{tab:criteria:Factory}
\end{table}

\subsection{Bewertung von Services}
\begin{table}[ht]
    \centering
    \begin{tabular}{|m{0.8\textwidth}|c|}   \hline
        \textbf{Kriterien}                                          & \textbf{Relevanz} \\ \hline\hline
        Service-Modul: \glqq application.<domain>\grqq{}	        & MINOR             \\ \hline
        Application Service-Name beinhaltet \glqq Application\grqq{}& INFO              \\ \hline
        Application Service-Modul: \glqq application\grqq{}	        & MINOR             \\ \hline
    \end{tabular}
    \caption{Kriterien zur Bewertung von Services}
    \label{tab:criteria:Service}
\end{table}

\section{Entwicklung einer Metrik zur Bewertung}


\section{Implementierung der Architekturanalyse}
\cite{DDD:Vernon:2013}

