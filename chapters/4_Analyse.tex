\chapter{Analyse der Architektur eines Java Systems}
\label{ch:analyse}
Diese Kapitel behandelt die Umsetzung der Architekturanalyse in Verbindung mit der Entwicklung einer Metrik zur Architekturbewertung. Im ersten Schritt wird darauf eingegangen anhand welcher Merkmale Domain-Driven Design Bausteine erkannt wurden. Danach wird der Schwerpunkt auf die Bewertung der eingelesenen Bausteine sowie der zugehörigen Architektur gelegt. Als letztes werden die einzelnen Bewertungen zu einer Metrik zusammengefasst, um den Zustand des eingelesenen Systems zu bewerten.

\section{Erkennung der Domain-Driven Design Artefakte}
\label{sec:analyse:characteristics}

\cite{DDD:Vernon:2013}
\cite{DDD:Vernon:2017}

\cite{DDD:Holmstrom:2019}
\cite{DDD:Le:2016}
\cite{DDD:Rademacher:2018a}
\cite{DDD:Hippchen:2017}
\cite{DDD:Nair:2019}

\subsection{Entities}
\label{sec:analyse:characteristics:entity}

\begin{figure}[htbp] 
    \centering
    \includegraphics[width=0.3\textwidth]{gfx/Characteristics-Entity.pdf}
    \caption{Beispiel für eine Entity}
    \label{fig:analyse:characteristics:entity}
\end{figure}

\subsection{Value Objects}
\label{sec:analyse:characteristics:value_object}

\begin{figure}[htbp] 
    \centering
    \includegraphics[width=0.3\textwidth]{gfx/Characteristics-Value_Object.pdf}
    \caption{Beispiel für ein Value Object}
    \label{fig:analyse:characteristics:value_object}
\end{figure}

\subsection{Aggregate Roots}
\label{sec:analyse:characteristics:aggregate}

\begin{figure}[htbp] 
    \centering
    \includegraphics[width=0.7\textwidth]{gfx/Characteristics-Aggregate_Root.pdf}
    \caption{Beispiel für eine Aggregate Root}
    \label{fig:analyse:characteristics:aggregate}
\end{figure}

\subsection{Domain Events}
\label{sec:analyse:characteristics:event}

\begin{figure}[htbp] 
    \centering
    \includegraphics[width=0.3\textwidth]{gfx/Characteristics-Domain_Event.pdf}
    \caption{Beispiel für ein Domain Event}
    \label{fig:analyse:characteristics:event}
\end{figure}

\subsection{Factories}
\label{sec:analyse:characteristics:factory}

\subsection{Repositories}
\label{sec:analyse:characteristics:repository}

\subsection{Services}
\label{sec:analyse:characteristics:service}

\subsection{Domain Services}
\label{sec:analyse:characteristics:service:domain}

\begin{figure}[htbp] 
    \centering
    \includegraphics[width=0.3\textwidth]{gfx/Characteristics-Domain_Service.pdf}
    \caption{Beispiel für ein Domain Service}
    \label{fig:analyse:characteristics:domain}
\end{figure}

\subsection{Application Services}
\label{sec:analyse:characteristics:service:application}

\begin{figure}[htbp] 
    \centering
    \includegraphics[width=0.3\textwidth]{gfx/Characteristics-Application_Service.pdf}
    \caption{Beispiel für ein Application Service}
    \label{fig:analyse:characteristics:application}
\end{figure}

\section{Bewertung der Domain-Driven Design Artefakte}
\label{sec:analyse:rating}
In diesem Abschnitt wird darauf eingegangen anhand von welchen Kriterien die Artefakte bewertet werden. Ein Kriterium besteht aus zwei Bestandteilen. Einer Beschreibung was erfüllt sein muss, um das Kriterium zu erfüllen und der Relevanz des Kriteriums. Die Relevanz hat einen direkten Einfluss auf die Gewichtung des Kriteriums.

\begin{table}[htbp] 
    \centering
    \begin{tabular}{|l||c|c|c|c|c|}   \hline
        \textbf{Einstufungen}   & \texttt{INFO} & \texttt{MINOR}    & \texttt{MAJOR}    & \texttt{CRITICAL}     & \texttt{BLOCKER}  \\ \hline
        \textbf{Gewichtung}     & 0             & 1                 & 3                 & 5                     & 10                \\ \hline
    \end{tabular}
    \caption{Einstufungen der Relevanz eines Kriteriums}
    \label{tab:analyse:rating:criteria}
\end{table}

In der Tabelle \ref{tab:analyse:rating:criteria} sind die fünf Einstufungen der Relevanz aufgeführt. Jeder Einstufung ist eine Gewichtung zugeordnet. Zum Beispiel sind Kriterien die als \texttt{INFO} gekennzeichnet sind nicht relevant für die Bewertung. Diese Kriterien beschreiben häufig die Benennung von Komponenten. 

Als \texttt{MINOR} oder \texttt{MAJOR} eingestufte Kriterien sollten eingehalten werden, aber von ihrer Einhaltung hängt nicht die komplette Architektur ab. Dies betrifft zum Beispiel das implementieren von vorgegebenen Methoden.

Kriterien welche hingegen als \texttt{CRITICAL} oder \texttt{BLOCKER} gekennzeichnet wurden, sollten zwingend erfüllt sein, da sonst Grundprinzipien der Architektur verletzt werden. Deshalb werden diese Kriterien besonders stark gewichtet.

\subsection{Bewertung von Modulen}
\label{sec:analyse:rating:modul}
Bei der Bewertung von Modulen muss man zwischen drei Kategorien von Modulen unterscheiden. Das wichtigste Modul im Domain-Driven Design ist das Domain-Modul. Denn im Domain-Modul sind alle Domain-Artefakte enthalten. Das zweitwichtigste Modul ist das Applikation-Modul. Das Applikation-Modul beinhaltet nahezu die gesamte Geschäftslogik des Systems. Das dritte Modul ist das Infrastruktur-Modul. In diesem Modul sind Schnittstellen implementiert, zum Beispiel für Datenbanken oder Benutzeroberflächen.

\begin{table}[htbp] 
    \centering
    \begin{tabular}{|m{0.8\textwidth}|c|}   \hline
        \textbf{Kriterien}                                          & \textbf{Relevanz} \\ \hline\hline
        Submodul von \glqq domain\grqq{}                            & \texttt{MINOR}    \\ \hline
        Beinhaltet nur Domain-Artifakte                             & \texttt{MAJOR}    \\ \hline
        Beinhaltet Aggregate Root                                   & \texttt{CRITICAL} \\ \hline
    \end{tabular}
    \caption{Kriterien zur Bewertung von Domain-Modulen}
    \label{tab:analyse:rating:modul:domain:criteria}
\end{table}

In der Tabelle \ref{tab:analyse:rating:modul:domain:criteria} sind die Kriterien von Domain-Modulen aufgelistet. Das erste Kriterium beschreibt, dass jedes Domain-Modul dem Modul \glqq domain\grqq{} untergeordnet sein sollte. Dieses Kriterium wurde als \texttt{MINOR} eingestuft, da es als Best Practice gilt, aber nicht zwingend notwendig ist.

Das zweite Kriterium betrachtet den Inhalt des Moduls, denn in einem Domain-Modul sollten nur Domain-Artefakte enthalten sein und keine Geschäfts- oder Infrastruktur-Logik. Da dieses Kriterium wichtiger ist, als das erste wurde es als \texttt{MAJOR} gekennzeichnet.

Das letzte Kriterium des Domain-Moduls, aus der Tabelle \ref{tab:analyse:rating:modul:domain:criteria}, ist gleichzeitig das wichtigste. Denn jedes Domain-Modul symbolisiert eine Domain und muss daher genau eine Aggregate Root beinhalten. Da dieses Kriterium für die korrekte Umsetzung von Domain-Driven Design notwendig ist, wurde es als \texttt{CRITICAL} eingestuft.

\begin{table}[htbp] 
    \centering
    \begin{tabular}{|m{0.8\textwidth}|c|}   \hline
        \textbf{Kriterien}                                          & \textbf{Relevanz} \\ \hline\hline
        Submodul von \glqq application\grqq{}                       & \texttt{MINOR}    \\ \hline
        Beinhaltet nur Application-Artifakte                        & \texttt{MAJOR}    \\ \hline
    \end{tabular}
    \caption{Kriterien zur Bewertung von Applikation-Modulen}
    \label{tab:analyse:rating:modul:applictaion:criteria}
\end{table}

\begin{table}[htbp] 
    \centering
    \begin{tabular}{|m{0.8\textwidth}|c|}   \hline
        \textbf{Kriterien}                                          & \textbf{Relevanz} \\ \hline\hline
        Submodul von \glqq infrastructure\grqq{}                    & \texttt{MINOR}    \\ \hline
        Beinhaltet nur Infrastructure-Artifakte                     & \texttt{MAJOR}    \\ \hline
    \end{tabular}
    \caption{Kriterien zur Bewertung von Infrastruktur-Modulen}
    \label{tab:analyse:rating:modul:infrastructure:criteria}
\end{table}

Die Kriterien für Applikation- und Infrastruktur-Module, aus den Tabellen \ref{tab:analyse:rating:modul:applictaion:criteria} und \ref{tab:analyse:rating:modul:infrastructure:criteria}, sind den ersten zwei Kriterien des Domain-Moduls sehr ähnlich.

\subsection{Bewertung von Domain Artefakten}
\label{sec:analyse:artifakt}
Der Schwerpunkt bei der Bewertung einer Softwarearchitektur auf Basis von Domain-Driven Design liegt darin, die einzelnen Domain Artefakte zu bewerten. Da die Domain Artefakte komplexer aufgebaut sind als Module, müssen sie auch anhand von mehr Kriterien bewertet werden.

\subsubsection{Bewertungskriterien von Entities}
\label{sec:analyse:rating:artifakt:entity}
\begin{table}[htbp] 
    \centering
    \begin{tabular}{|m{0.8\textwidth}|c|}   \hline
        \textbf{Kriterien}                                          & \textbf{Relevanz} \\ \hline\hline
        Modul: \glqq domain.<domain>.model\grqq{}                   & \texttt{MINOR}    \\ \hline
        Attribut-Typen: Entities oder Value Objects                 & \texttt{MAJOR}    \\ \hline
        Ein Attribut für die eigene Identität                       & \texttt{CRITICAL} \\ \hline
        Methoden \texttt{equals()} und \texttt{hashCode()}          & \texttt{MAJOR}    \\ \hline
        Getter für jedes Attribut                                   & \texttt{MINOR}    \\ \hline
        Setter für jedes Attribut                                   & \texttt{MINOR}    \\ \hline
    \end{tabular}
    \caption{Kriterien zur Bewertung von Entities}
    \label{tab:analyse:rating:artifakt:entity:criteria}
\end{table}

Die erste Tabelle \ref{tab:analyse:rating:artifakt:entity:criteria} zeigt die Kriterien, welche von allen Entities eingehalten werden müssen. Das erste Kriterium beschreibt welchem Modul Entities zugeordnet sind. Da Entities zur Modellierung von Domains eingesetzt werden, sollten sie sich in dem Modul \glqq model\grqq{} befinden \cite{DDD:Vernon:2013}.

Die Attribute von Entities sollten immer Instanzen von anderen Entities oder Value Objects sein. Zum Beispiel werden zum Modellieren von Identitäten Value Objects verwendet \cite{DDD:Vernon:2013}. Da dieses Kriterium einen wichtigen Bestandteil von Entities einnimmt, wurde es als \texttt{MAJOR} eingestuft.

In der dritten Zeile der Tabelle \ref{tab:analyse:rating:artifakt:entity:criteria} ist das wichtigste Kriterium einer Entity aufgeführt. Dieses Kriterium sagt aus, dass jede Entity, wie bereits in Abschnitt \ref{sec:basics:ddd:mdd:entity} beschrieben, eine Identität besitzen muss. Für dieses Kriterium ist die Art und Weise der Modellierung nicht relevant. Da dieses Kriterium der hauptsächliche Unterschied zu Value Objects ist \cite{DDD:Vernon:2013}, wurde es als \texttt{CRITICAL} gekennzeichnet.

Als letztes werden notwendige Methoden aufgeführt welche implementiert sein müssen. Dazu zählen Getter und Setter, also Methoden zum Auslesen und Setzten von Attributen \cite{DDD:Vernon:2013}. Die Existenz von Gettern und Settern wird mit \texttt{MINOR} gewichtet. Wichtiger für Entities ist die Existenz von den Methoden \texttt{equals()} und \texttt{hashCode()}. Die Implementierung dieser Methoden ermöglicht es Entities anhand ihrer Identität und nicht ihrer sonstigen Attribute zu unterscheiden \cite{DDD:Vernon:2013}. Da die Methoden \texttt{equals()} und \texttt{hashCode()} für die Funktionsweise von Entities wichtig sind, wurde dieses Kriterium als \texttt{MAJOR} eingestuft.

\subsubsection{Bewertungskriterien von Value Objects}
\label{sec:analyse:rating:artifakt:value_object}
\begin{table}[htbp] 
    \centering
    \begin{tabular}{|m{0.8\textwidth}|c|}   \hline
        \textbf{Kriterien}                                          & \textbf{Relevanz} \\ \hline\hline
        Modul: \glqq domain.<domain>.model\grqq{}                   & \texttt{MINOR}    \\ \hline
        Attribut-Typen: Value Objects oder Standard Datentypen      & \texttt{MAJOR}    \\ \hline
        Kein Attribut für eigenen Identität                         & \texttt{BLOCKER}  \\ \hline
        Methoden \texttt{equals()} und \texttt{hashCode()}          & \texttt{MAJOR}    \\ \hline
        Getter für jedes Attribut                                   & \texttt{MINOR}    \\ \hline
        Getter heißen so wie das zugehörige Attribut                & \texttt{INFO}     \\ \hline
        Setter für jedes Attribut                                   & \texttt{MINOR}    \\ \hline
        Setter ist frei von Seiteneffekten.                         & \texttt{CRITICAL} \\ \hline
    \end{tabular}
    \caption{Kriterien zur Bewertung von Value Objects}
    \label{tab:analyse:rating:artifakt:value_object:criteria}
\end{table}

Die Kriterien von Value Objects aus Tabelle \ref{tab:analyse:rating:artifakt:value_object:criteria} sind den Kriterien der Entities sehr ähnlich, unterscheiden sich aber teilweise deutlich. Das erste unterschiedliche Kriterium steht in der zweiten Zeile. Das Kriterium sagt aus, dass Value Objects nur Attribute von anderen Value Objects oder Java Standard Datentypen besitzen darf.

Das erste Kriterium, von Value Objects, welches sich grundlegend unterscheidet, steht in der dritten Zeile der Tabelle \ref{tab:analyse:rating:artifakt:value_object:criteria}. Wie bereits in Abschnitte \ref{sec:basics:ddd:mdd:value_object} beschrieben, stellt die Existenz einer Identität eine Architekturverletzung dar. Deshalb wurde dieses Kriterium als \texttt{BLOCKER} eingestuft.

Auch ein Value Object sollte Getter und Setter für jedes Attribut besitzen. Allerdings sind dabei einzelne Kriterien zu beachten. Zur besseren Lesbarkeit wird empfohlen Getter, ohne den Präfix \glqq get \grqq{} zu benennen \cite{DDD:Vernon:2013}. Da die Verletzung dieses Kriteriums nicht gefährlich ist, wurde es als \texttt{INFO} gekennzeichnet.

Bei den Settern hingegen muss beachtet werden, dass sie keine Seiteneffekte haben \cite{DDD:Evans:2003}. Dieses Kriterium wird erfüllt, wenn die Methode als \texttt{private} gekennzeichnet wurde oder wenn die Methode eine neue Instanz von dem Value Objects zurückliefert. In beiden Fällen wird gewährleistet, dass die Werte des Objects nicht verändert werden. Da die Unveränderbarkeit der Value Objecte eine Grundvoraussetzung ist dieses Kriterium \texttt{CRITICAL}.

\subsubsection{Bewertungskriterien von Aggregate Roots}
\label{sec:analyse:rating:artifakt:aggregate}
\begin{table}[htbp] 
    \centering
    \begin{tabular}{|m{0.8\textwidth}|c|}   \hline
        \textbf{Kriterien}                                          & \textbf{Relevanz} \\ \hline\hline
        Modul: \glqq domain.<domain>.model\grqq{}                   & \texttt{MINOR}    \\ \hline
        Attribut-Typen: andere Entities oder Value Objects          & \texttt{MAJOR}    \\ \hline
        Ein Attribut für die eigene Identität                       & \texttt{CRITICAL} \\ \hline
        Methoden \texttt{equals()} und \texttt{hashCode()}          & \texttt{MAJOR}    \\ \hline
        Getter für jedes Attribut                                   & \texttt{MINOR}    \\ \hline
        Setter für jedes Attribut                                   & \texttt{MINOR}    \\ \hline
        Repository für die Aggregate Root existiert                 & \texttt{MAJOR}    \\ \hline
        Factory für die Aggregate Root existiert                    & \texttt{MAJOR}    \\ \hline
        Service für die Aggregate Root existiert                    & \texttt{MAJOR}    \\ \hline
    \end{tabular}
    \caption{Kriterien zur Bewertung von Aggregate Roots}
    \label{tab:analyse:rating:artifakt:aggregate:criteria}
\end{table}

Auch die Kriterien von Aggregate Roots, in der Tabelle \ref{tab:analyse:rating:artifakt:aggregate:criteria}, sind den Kriterien von Entities sehr ähnlich. Da Aggregate Roots, wie bereits in Abschnitt \ref{sec:basics:ddd:mdd:aggregate} beschrieben, nur spezielle Entities sind, sind die ersten sechs Kriterien identisch.

Zusätzlich zu den ersten sechs Kriterien besitzt die Aggregate Root drei weitere. Da eine Aggregate Root ihre zugeordnete Domain repräsentiert, werden zur Verwaltung drei Komponenten benötigt. Nämlich eine Repository, eine Factory und mindestens einen Service \cite{DDD:Vernon:2013}. Diese drei Kriterien wurden aufgrund ihrer Wichtigkeit als \texttt{MAJOR} eingestuft.

\subsubsection{Bewertungskriterien von Domain Events}
\label{sec:analyse:rating:artifakt:event}
\begin{table}[htbp]
    \centering
    \begin{tabular}{|m{0.8\textwidth}|c|}   \hline
        \textbf{Kriterien}                                          & \textbf{Relevanz} \\ \hline\hline
        Modul: \glqq domain.<domain>.model\grqq{}                   & \texttt{MINOR}    \\ \hline
        Attribut für Zeitstempel                                    & \texttt{MAJOR}    \\ \hline
        Attribut für die ID einer Entity                            & \texttt{MAJOR}    \\ \hline
    \end{tabular}
    \caption{Kriterien zur Bewertung von Domain Events}
    \label{tab:analyse:rating:artifakt:event:criteria}
\end{table}

Das erste Kriterium, aus der Tabelle \ref{tab:analyse:rating:artifakt:event:criteria}, von Domain Events verhält sich analog zu den bereits behandelten Komponenten. 

Wie bereits im Abschnitt \ref{sec:basics:ddd:mdd:event} beschrieben wurde, muss ein Domain Event zwei Attribute besitzen. Ein Attribut welches den Zeitpunkt des Auftretens des Events dokumentiert. Sowie ein Attribut mit der Identität der beteiligten Entity. Die Relevanz beider Kriterien ist \texttt{MAJOR}.

\subsubsection{Bewertungskriterien von Factories}
\label{sec:analyse:rating:artifakt:factory}
\begin{table}[htbp] 
    \centering
    \begin{tabular}{|m{0.8\textwidth}|c|}   \hline
        \textbf{Kriterien}                                      & \textbf{Relevanz} \\ \hline\hline
        Interface-Namenssuffix: \glqq Factory\grqq{}            & \texttt{INFO}     \\ \hline 
        Klassen-Namenssuffix: \glqq FactoryImpl\grqq{}          & \texttt{INFO}     \\ \hline
        Interface Modul: \glqq domain.<domain>.model\grqq{}     & \texttt{MINOR}    \\ \hline
        Klassen Modul: \glqq domain.<domain>.model.impl\grqq{}  & \texttt{MINOR}    \\ \hline
        Factory-Klasse implementiert Factory-Interface          & \texttt{MINOR}    \\ \hline
        Enthält eine Instanz eines Repositories	                & \texttt{MAJOR}    \\ \hline
        Methode zum Erzeugen (\texttt{create})                  & \texttt{MAJOR}    \\ \hline
    \end{tabular}
    \caption{Kriterien zur Bewertung von Factories}
    \label{tab:analyse:rating:artifakt:factory:criteria}
\end{table}

Bei der Bewertung von Factories muss zunächst zwischen Factory-Interfaces und Factory-Klassen unterschieden werden. Denn es gibt für sie unterschiedliche Namenskonventionen und sie sollten in unterschiedlichen Modulen untergebracht sein. Die Namenskonventionen in den zwei ersten Zeilen der Tabelle \ref{tab:analyse:rating:artifakt:factory:criteria} sind als \texttt{INFO} gekennzeichnet da es eine Java-Konvention ist Klassen welche ein Interface implementieren mit dem Suffix \glqq Impl\grqq{} zu benennen \cite{DDD:Vernon:2013}.

Die Factory-Interfaces und Factory-Klassen unterscheiden sich aber nicht nur bei ihrer Namenskonventionen, sondern auch bei den Modulen denen sie zugeordnet sind. Während nämlich die Interfaces ein Bestandteil des Moduls \glqq model\grqq{} sind, sollten die Klassen in einem untergeordneten Modul, mit dem Namen \glqq impl\grqq{} abgelegt sein \cite{DDD:Vernon:2013}. Ähnlich wie bei den vorherigen Artefakten ist die Zuordnung zu Modulen als \texttt{MINOR} eingestuft.

Da eine Factory, wie im Abschnitt \ref{sec:basics:ddd:mdd:factory} beschrieben, für die Erzeugung von Entities zuständig ist, benötigt sie eine Instanz des dazugehörigen Repositories, um die neue Entity zu persistieren. Das Erzeugen der Entity sollte, wie in der letzten Zeile der Tabelle \ref{tab:analyse:rating:artifakt:factory:criteria} beschrieben, mit der Methode \texttt{create()} geschehen. Diese beiden Kriterien wurden mit einer Relevanz von \texttt{MAJOR} bewertet.

\subsubsection{Bewertungskriterien von Repositories}
\label{sec:analyse:rating:artifakt:repository}
\begin{table}[htbp] 
    \centering
    \begin{tabular}{|m{0.8\textwidth}|c|}   \hline
        \textbf{Kriterien}                                                  & \textbf{Relevanz} \\ \hline\hline
        Interface-Namenssuffix: \glqq Repository\grqq{}                     & \texttt{INFO}     \\ \hline 
        Klassen-Namenssuffix: \glqq RepositoryImpl\grqq{}                   & \texttt{INFO}     \\ \hline
        Interface Modul: \glqq domain.<domain>.model\grqq{}                 & \texttt{MINOR}    \\ \hline
        Klassen Modul: \glqq domain.<domain>.model.impl\grqq{}              & \texttt{MINOR}    \\ \hline
        Repository-Klasse implementiert Repository-Interface                & \texttt{MINOR}    \\ \hline
        Methode zum Erzeugen einer neuen Identität (\texttt{nextIdentity})  & \texttt{MAJOR}    \\ \hline
        Methode zum Finden (\texttt{findBy/get})                            & \texttt{MAJOR}    \\ \hline
        Methode zum Speichern (\texttt{save/add/insert/put})                & \texttt{MAJOR}    \\ \hline
        Methode zum Löschen (\texttt{delete/remove})                        & \texttt{MAJOR}    \\ \hline
        Methode zum Überprüfen der Existenz (\texttt{contains/exists})      & \texttt{MINOR}    \\ \hline
        Methode zum Aktualisieren (\texttt{update})                         & \texttt{MINOR}    \\ \hline
    \end{tabular}
    \caption{Kriterien zur Bewertung von Repositories}
    \label{tab:analyse:rating:artifakt:repository:criteria}
\end{table}

Die ersten fünf Kriterien von Repositories, in der Tabelle \ref{tab:analyse:rating:artifakt:repository:criteria} sind gleich aufgebaut wie im vorherigen Abschnitt \ref{sec:analyse:rating:artifakt:factory}.

Bei den notwendigen Methoden unterscheiden sich allerdings die Kriterien der Repositories und der Factories. Die erforderlichen Methoden wurden in zwei Kategorien eingeteilt.

Die erste Kategorie enthält Methoden, welche unverzichtbar für ein Repository sind. Zu den Methoden der ersten Kategorie gehören Methoden zum Erzeugen einer neuen Identität sowie zum Finden, Speichern und Löschen einer Entity. Die Methoden dieser Kategorie wurden als \texttt{MAJOR} eingestuft.

In der zweiten Kategorie sind Methoden, welche nützlich sind aber nicht so relevant wie die Methoden aus der ersten Kategorie, zum Beispiel Methoden zum Aktualisieren und Überprüfen der Existenz einer Entity. Aufgrund ihrer geringeren Relevanz wurden diese Methoden als \texttt{MINOR} bewertet.

\subsubsection{Bewertungskriterien von Services}
\label{sec:analyse:rating:artifakt:service}
\begin{table}[htbp] 
    \centering
    \begin{tabular}{|m{0.8\textwidth}|c|}   \hline
        \textbf{Kriterien}                                          & \textbf{Relevanz} \\ \hline\hline
        Application Service Name beinhaltet \glqq Application\grqq{}& \texttt{INFO}     \\ \hline
        Application Service Modul: \glqq application\grqq{}	        & \texttt{MINOR}    \\ \hline
        Domain Service Modul: \glqq application.<domain>\grqq{}	    & \texttt{MINOR}    \\ \hline
    \end{tabular}
    \caption{Kriterien zur Bewertung von Services}
    \label{tab:analyse:rating:artifakt:service:criteria}
\end{table}

Bei der Bewertung von Services muss zwischen zwei Services unterschieden werden, die Application und Domain Services. 

Die Application Services werden anhand von zwei Kriterien bewertet, siehe Tabelle \ref{tab:analyse:rating:artifakt:service:criteria}. Der Name des Application Service sollte das Wort \glqq Application\grqq{} beinhalten. Dieses Kriterium ist mit \texttt{INFO} gewichtet. Außerdem muss der Application Service dem Modul \glqq application\grqq{} zugeordnet sein. Die Zuordnung zu dem speziellen Modul ist als \texttt{MINOR} eingestuft.

Die Domain Services erfüllen alle Kriterien wenn, sie in dem Modul \glqq application.<domain>\grqq{} enthalten sind. Auch dieses Kriterium besitzt eine Relevanz von \texttt{MINOR}.

\subsection{Dokumentation der nicht erfüllten Kriterien}
\label{sec:analyse:rating:issue}
Damit die Bewertungen der Domain-Driven Design Baustein zu jedem Zeitpunkt nachvollzogen werden kann, werden die nicht erfüllten Kriterien zurückgegeben.

\begin{lstlisting}[caption={Ausschnitt der Bewertung eines Artefakts},captionpos=b,label=lst:analyse:rating:issue:output]
{
    "DDD": "VALUE_OBJECT",
    "domain": "issue",
    "name": "IssueId",
    "issues": [
        {
            "description": "The Value Object 'IssueId' is not placed at 'domain.issue.model'",
            "type": "MINOR"
        },
        ...
    ]
}
\end{lstlisting}

Listing \ref{lst:analyse:rating:issue:output} zeigt einen Ausschnitt des Rückgabewertes eines bewerteten Artefakts. Man kann daran deutlich erkennen, welche Kriterien nicht eingehalten wurden und woran es gelegen hat.

\section{Bewertung anhand einer Metrik}
\label{sec:analyse:metric}
Damit eine Aussage über den Zustand des Gesamtsystems getroffen werden kann, ist es notwendig eine Metrik zu definieren. In dieser Metrik sollten alle Kriterien und ihre Relevanz einbezogen werden, damit sie aussagekräftig ist.

\begin{equation}
    \text{Totale Kriterien} = \sum \limits_{k}^\text{Kriterien} Relevanz(k) \label{align:analyse:metric:total}
\end{equation}

m ersten Schritt muss, anhand der Formel \ref{align:analyse:metric:total}, die Wertigkeit aller Kriterien berechnet werden. Dafür werden die Gewichtungen aller benötigten Kriterien summiert. Dadurch erhält man die maximal mögliche Bewertung des Systems.

\begin{equation}
    \text{Erfüllte Kriterien} = \sum \limits_{k}^\text{Kriterien} Relevanz(k)\text{ falls }k\text{ erfüllt} \label{align:analyse:metric:fulfilled}
\end{equation}

Als Nächstes wird die Formel \ref{align:analyse:metric:fulfilled} verwendet, um die Summe der Gewichtungen aller erfüllten Kriterien zu erhalten. Dieser Wert symbolisiert den tatsächlichen Zustand des Systems.

\begin{equation}
    \text{Metrik} = \frac{\text{Erfüllte Kriterien} \cdot 100}{\text{Totale Kriterien}} \label{align:analyse:metric}
\end{equation}

Zur Berechnung der Gesamtbewertung werden die vorherigen Ergebnisse mit der Formel \ref{align:analyse:metric} miteinander verrechnet. Dabei erhält man eine Metrik, welche die prozentuale Erfüllung aller benötigten Kriterien, unter Einbeziehung ihrer Relevanz, aussagt.

\begin{table}[htbp] 
    \centering
    \begin{tabular}{|c|l|l|}   \hline
        \textbf{Note}   & \textbf{Beschreibung} & \textbf{Wertebereich}                     \\ \hline\hline
        \texttt{A}      & sehr gut              & $\text{Metrik } \geq 90.0$                \\ \hline
        \texttt{B}      & gut                   & $90.0 \gneqq \text{ Metrik } \geq 80.0$   \\ \hline
        \texttt{C}      & befriedigend          & $80.0 \gneqq \text{ Metrik } \geq 60.0$   \\ \hline
        \texttt{D}      & ausreichend           & $60.0 \gneqq \text{ Metrik } \geq 40.0$   \\ \hline
        \texttt{E}      & mangelhaft            & $40.0 \gneqq \text{ Metrik } \geq 20.0$   \\ \hline
        \texttt{F}      & ungenügend            & $20.0 \gneqq \text{ Metrik } \geq 0.0$    \\ \hline
    \end{tabular}
    \caption{Benotung der Domain-Driven Design Metrik}
    \label{tab:analyse:metric}
\end{table}

Um die Metrik aus \ref{align:analyse:metric} besser einzuordnen, wird dem System anhand der Tabelle \ref{tab:analyse:metric} eine Note gegeben.

\begin{lstlisting}[caption={Ausschnitt der Gesamtbewertung eines Systems},captionpos=b,label=lst:analyse:metric:output]
{
    "score": "D",
    "criteria": {
        "total": 121,
        "fulfilled": 64
    },
    "fitness": 52.89,
    "#Issues": 35
}
\end{lstlisting}

Anschließend wird die Gesamtbewertung in dem Format aus Listing \ref{lst:analyse:metric:output} zurückgegeben.