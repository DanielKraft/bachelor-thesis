\chapter{Zusammenfassung und Ausblick}
\label{ch:summary}
Dieses Kapitel zeigt die Ergebnisse der Arbeit auf und geht in einem Ausblick auf zukünftige Forschungsansätze ein.

\section{Ergebnisse dieser Arbeit}
\label{sec:summary}
In \autoref{ch:analyse}, wurde gezeigt, dass es möglich ist mit Hilfe von jQAssistant Java-Systeme zu analysieren und den verschiedenen Artefakten den Domain-Driven Design Bausteinen zuzuweisen. Dabei wurde auf verschiedene Problemstellungen eingegangen und erläutert, wie diese gelöst wurden. Für die Zuweisung wurden für alle Domain-Driven Design Bausteine Erkennungsmerkmale definiert und anschließend in dem Analysealgorithmus umgesetzt. Dass der Analysealgorithmus alle Domain-Driven Design Bausteinen erkennen und zuweisen kann, wurde anhand der drei Beispielsysteme im \autoref{ch:anhang:beispiel} gezeigt.

Zusätzlich zu einer bestehenden Metrik wurde im Rahmen der Arbeit eine eigene Metrik entwickelt. Beide Metriken wurden im \autoref{ch:rating} näher erläutert. Die bestehende Metrik wurde zur Bewertung des Object-Oriented Design verwendet. Die neu entwickelte Metrik bewertet, wie gut ein Java-System die Kriterien vom Domain-Driven Design einhält. Dafür wurde für jeden Domain-Driven Design Baustein Bewertungskriterien definiert und gewichtet. Um dem Nutzer einen schnellen Überblick über den Zustand des Java-Systems zu ermöglichen, wurde ein Notensystem eingeführt. Zusätzlich zu der Gesamtbewertung hat der Nutzer die Möglichkeit, sich die Bewertung jedes Artefaktes anzuzeigen. Dafür wurde während der Bewertung für nicht erfüllten Bewertungskriterien Hinweise erzeugt, welche dem Nutzer beim Verständnis der Bewertung helfen sollen. Durch diese Hinweise ist der Nutzer in der Lage, das System ohne weitere Hilfe zu verbessern.

Um dem Nutzer die Aufgabe, das Java-System zu verbessern, abzunehmen, wurde ein Umwandlungsalgorithmus entwickelt. Die Umsetzung dieses Umwandlungsalgorithmus wurde im \autoref{ch:suggestion} beschrieben und bildet den Schwerpunkt der Arbeit. Der Umwandlungsalgorithmus ist in der Lage sämtliche Java-Systeme anhand von Umwandlungskriterien in das Domain-Driven Design zu überführen. Die Umwandlungskriterien wurden von den Bewertungskriterien abgeleitet. Nachdem der Umwandlungsalgorithmus auf die Beispielsysteme angewendet wurde, konnte mithilfe der Metriken gezeigt werden, dass der Umwandlungsalgorithmus in der Lage ist, Java-Systeme aller Art in das Domain-Driven Design zu überführen, auch wenn nicht immer alle Kriterien erfüllt werden konnten.

\section{Ausblick in zukünftige Arbeiten}
\label{sec:summary:future}
Es gibt viele Möglichkeiten auf den Ergebnissen dieser Arbeit aufzubauen. Ein erster Ansatz könnte sein, die Problemszenarien aus dem \autoref{sec:discuss:limits} umzusetzen, um die Qualität des Verbesserungsvorschlags zu erhöhen. Eine solche Erweiterung muss in der Lage sein, sinngemäße Zusammenhänge zwischen Artefakten zu erkennen.

Ein weiterer Ansatz wäre es, die Domain-Driven Design Metrik genauer zu spezifizieren, indem man die Bewertungskriterien aus \autoref{sec:rating} erweitert. Dadurch könnte die Metrik auch außerhalb des Kontextes dieser Arbeit verwendet werden.

Das Entwickeln eines Umwandlungsalgorithmus ist nicht nur auf das Domain-Driven Design beschränkt. Für einen Entwickler könnte es nützlich sein sich mehrere Verbesserungsvorschläge auf Basis verschiedener Softwarearchitekturen generieren zu lassen. Dadurch hätte der Entwickler die Möglichkeit, zwischen verschieden Architekturen zu wählen und die beste Architektur für sein System zu finden.

Weiterhin kann es hilfreich sein, solche Umwandlungsalgorithmen auch für andere objektorientierte Programmiersprachen umzusetzen.