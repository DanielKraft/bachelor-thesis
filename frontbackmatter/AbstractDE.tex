%*******************************************************
% Abstract in German
%*******************************************************
\begin{otherlanguage}{ngerman}
	\pdfbookmark[1]{Zusammenfassung}{Zusammenfassung}
	\chapter*{Zusammenfassung}
    
    Zurzeit wird gerne das Domain-Driven Design verwendet. Aber nicht in allen Systemen werden alle Standards des Domain-Driven Design eingehalten, obwohl das Domain-Driven Design dabei helfen kann die Qualität des Quellcodes zu verbessern.

    In dieser Arbeit werden die Analyse, die Bewertung, sowie die Verbesserung von Java-Systemen, basierend auf dem Domain-Driven Design, erläutert. Das Ziel dieser Arbeit ist es, Java-Systeme zu analysieren und einen Verbesserungsvorschlag im Domain-Driven Design zu generieren. Dabei ergeben sich folgende Problemstellungen. Zum einen müssen die verschieden Bausteine des Domain-Driven Design erkannt werden, damit der Umwandlungsalgorithmus eine Arbeitsgrundlage besitzt. Auch beim Verbessern müssen verschiedene Problemstellungen beachtet werden. Zum Beispiel das Erzeugen von neuen Artefakten.
    
    Zur Analyse von Java-Systemen wird das Analysewerkzeug jQAssistant verwendet und anschließend die Artefakte des Java-Systems anhand von ihren Merkmalen den Bausteinen zugeordnet. Zur Bewertung der Java-Systeme werden verschiedene Metriken benutzt. Eine der Metriken wird im Laufe dieser Arbeit entwickelt. Nachdem das ursprüngliche Java-System bewertet wurde, wird es mithilfe eines Umwandlungsalgorithmus verbessert. Anschließend wird der Verbesserungsvorschlag mithilfe der vorher eingeführten Metriken erneut bewertet.
    
    Anhand der Untersuchung mehrerer Beispielsysteme geht hervor, dass das entwickelte Werkzeug in der Lage ist für ein Java-System einen Verbesserungsvorschlag zu erzeugen, welcher besser bewertet wird als das ursprüngliche System. Daraus resultiert, dass es prinzipiell möglich ist die Architektur eines Java-Systems mithilfe eines Umwandlungsalgorithmus zu verbessern. Allerdings weisen die Verbesserungsvorschläge an vereinzelten Stellen Schwächen auf, welche nur durch manuelles Anpassen behoben werden können.
    
\end{otherlanguage}
