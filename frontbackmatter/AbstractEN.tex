%*******************************************************
% Abstract in English
%*******************************************************
\pdfbookmark[1]{Abstract}{Abstract}


\begin{otherlanguage}{american}
	\chapter*{Abstract}
	
    Currently, domain-driven design is widely used in software development. But not all systems meet all standards of Domain-Driven Design, although Domain-Driven Design can help to improve the quality of the source code. For example, domain-driven design can increase the modularity and readability of the source code.

    In this thesis, the analysis, evaluation, and improvement of Java systems based on Domain-Driven Design are explained. The goal of this work is to analyze Java systems and to generate a proposal for improvement in Domain-Driven Design. The following problems arise. On the one hand, the different building blocks of Domain-Driven Design have to be recognized so that the conversion algorithm has a working basis. On the other hand various problems must also be taken into account when making improvements. For example, the generation of new artefacts.
        
    The analysis tool jQAssistant is used to analyse Java systems and then the artefacts of the Java system are assigned to the building blocks based on their characteristics. Different metrics are used to evaluate the Java systems. One of the metrics will be developed in the course of this work. After the original Java system has been evaluated, it is improved using a conversion algorithm. Afterwards, the proposed improvement is re-evaluated using the previously introduced metrics.

    The analysis of several example systems shows that the developed tool is able to generate a suggestion for improvement of a Java system, which is rated better than the original system. As a result, it is in principle possible to improve the architecture of a Java system using a conversion algorithm. However, the improvement suggestions show weaknesses in isolated places, which can only be corrected by manual adjustment.


\end{otherlanguage}